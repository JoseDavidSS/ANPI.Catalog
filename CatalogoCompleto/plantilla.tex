\documentclass[10pt]{article}
%\usepackage[latin1]{inputenc}

\usepackage[spanish]{babel}
\usepackage[utf8]{inputenc}
\usepackage{amsmath,multicol,enumerate}
\usepackage{amsfonts}
\usepackage{multicol}





\usepackage{wrapfig}


%%%%% Formato REVISTA DE MATEMÁTICA: TEORÍA Y APLICACIONES%%%%%%%
\topmargin=-2cm\textheight=23cm\textwidth=19cm
\oddsidemargin=-1cm\evensidemargin=-1cm
%%%%%%%%%%%%%%%%%%%%%%%%%%%%%%%%%%%%%%%%%%%%%%%%%%%%%%%%%%%%%%%%%
\parskip=0.25cm
\parindent=0mm

\renewcommand{\figurename}{Figura}

\usepackage[dvips]{graphicx}
\usepackage{url}

%Utilizamos el paquete para incorporar graficos postcript


%\usepackage{amssymb}
\usepackage[psamsfonts]{amssymb} %paquetes para los simbolos matematicos
\usepackage{latexsym}

\usepackage{color}

\usepackage[framed]{matlab-prettifier}


\usepackage[T1]{fontenc}

\renewcommand{\lstlistingname}{C\'odigo}


\definecolor{mygray}{rgb}{0.4,0.4,0.4}
\definecolor{mygreen}{rgb}{0,0.8,0.6}
\definecolor{myorange}{rgb}{1.0,0.4,0}



%%%%% Para Insertar Lenguaje M (MATLAB/GNU OCTAVE)



\begin{document}

%Encabezado
Instituto Tecnológico de Costa Rica \hfill CE-3102: Análisis Numéricos para Ingeniería\\
Ingeniería en Computadores \hfill Semestre: II - 2021\\

\begin{center}\textbf{\huge Catálogo Grupal de Algoritmos}  \end{center}

{\bf Integrantes: }
\begin{itemize}
\item Jessica Espinoza Quesada - Carnet 2018135811
\item Jose David Sánchez Schnitzler - Carnet 2018142388
\item Tomás Felipe Segura Monge - Carnet 2018099729
\end{itemize}

\section{Tema 1: Ecuaciones No lineales}

\subsection{Método 1: Bisección}
\UseRawInputEncoding
%Lenguaje Python
\lstinputlisting[language=Python,
basicstyle=\ttfamily\normalsize,morekeywords={as,__init__,MyClass}, keywordstyle=\color{violet}\ttfamily, frame=single, commentstyle=\color{gray}\ttfamily, showstringspaces=false, caption={Biseccion.}]{parte1/biseccion.py}
\inputencoding{utf8}

\newpage

\subsection{Método 2: Newton-Raphson}
\UseRawInputEncoding
%Lenguaje Octave/Matlab
\lstinputlisting[style=Matlab-editor, basicstyle=\mlttfamily\normalsize, caption={Newton Raphson}]{parte1/newton_raphson.m}
\inputencoding{utf8}

\newpage

\subsection{Método 3: Secante}
\UseRawInputEncoding
%Lenguaje C++
\lstinputlisting[basicstyle=\normalsize\ttfamily\color{black}, commentstyle=\color{mygray}, frame=single, numbersep=5pt, numberstyle=\tiny\color{mygray}, keywordstyle=\color{mygreen}, showspaces=false, showstringspaces=false, stringstyle=\color{myorange}, tabsize=2, language=C++, caption={Metodo de la secante.}]{parte1/secante.cpp}
\inputencoding{utf8}

\newpage

\subsection{Método 4: Falsa Posición}
\UseRawInputEncoding
%Lenguaje C++
\lstinputlisting[basicstyle=\normalsize\ttfamily\color{black}, commentstyle=\color{mygray}, frame=single, numbersep=5pt, numberstyle=\tiny\color{mygray}, keywordstyle=\color{mygreen}, showspaces=false, showstringspaces=false, stringstyle=\color{myorange}, tabsize=2, language=C++, caption={Metodo de la falsa posicion.}]{parte1/falsa_posicion.cpp}
\inputencoding{utf8}

\newpage

\subsection{Método 5: Punto Fijo}
\UseRawInputEncoding
%Lenguaje Octave/Matlab
\lstinputlisting[style=Matlab-editor, basicstyle=\mlttfamily\normalsize, caption={Punto Fijo.}]{parte1/punto_fijo.m}
\inputencoding{utf8}

\newpage

\subsection{Método 6: Muller}
\UseRawInputEncoding
%Lenguaje Python
\lstinputlisting[language=Python,
basicstyle=\ttfamily\small,morekeywords={as,__init__,MyClass}, keywordstyle=\color{violet}\ttfamily, frame=single, commentstyle=\color{gray}\ttfamily, showstringspaces=false, caption={Muller.}]{parte1/muller.py}
\inputencoding{utf8}

\newpage

\section{Tema 2: Optimización}

\subsection{Método 1: Descenso Coordinado}
\UseRawInputEncoding
%Lenguaje Python
\lstinputlisting[language=Python,
basicstyle=\ttfamily\normalsize,morekeywords={as,__init__,MyClass}, keywordstyle=\color{violet}\ttfamily, frame=single, commentstyle=\color{gray}\ttfamily, showstringspaces=false, caption={coordinado}]{parte2/coordinado.py}
\inputencoding{utf8}

\newpage

\subsection{Método 2: Gradiente Conjugado no Lineal}
\UseRawInputEncoding
%Lenguaje Octave/Matlab
\lstinputlisting[style=Matlab-editor, basicstyle=\mlttfamily\normalsize, caption={gradiente}]{parte2/gradiente.m}
\inputencoding{utf8}

\newpage

\section{Tema 3: Sistemas de Ecuaciones}

\subsection{Métodos auxiliares}

\subsubsection{Determinante de matriz}

\UseRawInputEncoding
%Lenguaje Python
\lstinputlisting[language=Python,
basicstyle=\ttfamily\normalsize,morekeywords={as,__init__,MyClass}, keywordstyle=\color{violet}\ttfamily, frame=single, commentstyle=\color{gray}\ttfamily, showstringspaces=false, caption={determinante de matriz nxn}]{parte3/python/métodos auxiliares/determinanteNxN.py}
\inputencoding{utf8}

\newpage

\subsubsection{Sustitución hacia atrás}

\UseRawInputEncoding
%Lenguaje Python
\lstinputlisting[language=Python,
basicstyle=\ttfamily\normalsize,morekeywords={as,__init__,MyClass}, keywordstyle=\color{violet}\ttfamily, frame=single, commentstyle=\color{gray}\ttfamily, showstringspaces=false, caption={sustitucion hacia atras}]{parte3/python/métodos auxiliares/sust_atras.py}
\inputencoding{utf8}

\newpage

\subsubsection{Sustitución hacia adelante}

\UseRawInputEncoding
%Lenguaje Python
\lstinputlisting[language=Python,
basicstyle=\ttfamily\normalsize,morekeywords={as,__init__,MyClass}, keywordstyle=\color{violet}\ttfamily, frame=single, commentstyle=\color{gray}\ttfamily, showstringspaces=false, caption={sustitucion hacia adelante}]{parte3/python/métodos auxiliares/sust_adelante.py}
\inputencoding{utf8}

\newpage

\subsection{Método 1: Eliminación Gaussiana}

\UseRawInputEncoding
%Lenguaje Python
\lstinputlisting[language=Python,
basicstyle=\ttfamily\normalsize,morekeywords={as,__init__,MyClass}, keywordstyle=\color{violet}\ttfamily, frame=single, commentstyle=\color{gray}\ttfamily, showstringspaces=false, caption={eliminacion gaussiana}]{parte3/python/gaussiana.py}
\inputencoding{utf8}
\newpage


\subsection{Método 2: Factorización LU}

\UseRawInputEncoding
%Lenguaje C++
\lstinputlisting[basicstyle=\normalsize\ttfamily\color{black}, commentstyle=\color{mygray}, frame=single, numbersep=5pt, numberstyle=\tiny\color{mygray}, keywordstyle=\color{mygreen}, showspaces=false, showstringspaces=false, stringstyle=\color{myorange}, tabsize=2, language=C++, caption={Metodo de la factorización LU.}]{parte3/C++/LU.cpp}
\inputencoding{utf8}
\newpage

\UseRawInputEncoding
%Lenguaje C++
\lstinputlisting[basicstyle=\normalsize\ttfamily\color{black}, commentstyle=\color{mygray}, frame=single, numbersep=5pt, numberstyle=\tiny\color{mygray}, keywordstyle=\color{mygreen}, showspaces=false, showstringspaces=false, stringstyle=\color{myorange}, tabsize=2, language=C++, caption={Metodo auxiliar de la factorizacion LU para obtener el determinante de una matriz.}]{parte3/C++/determinant.h}
\inputencoding{utf8}
\newpage

\subsection{Método 3: Factorización de Cholesky}

\UseRawInputEncoding
%Lenguaje Python
\lstinputlisting[language=Python,
basicstyle=\ttfamily\normalsize,morekeywords={as,__init__,MyClass}, keywordstyle=\color{violet}\ttfamily, frame=single, commentstyle=\color{gray}\ttfamily, showstringspaces=false, caption={factorizacion de Cholesky}]{parte3/python/fact_cholesky.py}
\inputencoding{utf8}

\newpage


\subsection{Método 4: Método de Thomas.}

\UseRawInputEncoding
%Lenguaje C++
\lstinputlisting[basicstyle=\normalsize\ttfamily\color{black}, commentstyle=\color{mygray}, frame=single, numbersep=5pt, numberstyle=\tiny\color{mygray}, keywordstyle=\color{mygreen}, showspaces=false, showstringspaces=false, stringstyle=\color{myorange}, tabsize=2, language=C++, caption={Metodo de Thomas.}]{parte3/C++/thomas.cpp}
\inputencoding{utf8}
\newpage

\subsection{Método 5: Jacobi}

\UseRawInputEncoding
%Lenguaje Python
\lstinputlisting[language=Python,
basicstyle=\ttfamily\normalsize,morekeywords={as,__init__,MyClass}, keywordstyle=\color{violet}\ttfamily, frame=single, commentstyle=\color{gray}\ttfamily, showstringspaces=false, caption={jacobi}]{parte3/python/jacobi.py}
\inputencoding{utf8}

\newpage

\subsection{Método 6: Gauss-Seidel}

\UseRawInputEncoding
%Lenguaje Python
\lstinputlisting[language=Python,
basicstyle=\ttfamily\normalsize,morekeywords={as,__init__,MyClass}, keywordstyle=\color{violet}\ttfamily, frame=single, commentstyle=\color{gray}\ttfamily, showstringspaces=false, caption={gauss-seidel}]{parte3/python/gauss_seidel.py}
\inputencoding{utf8}

\newpage

\subsection{Método 7: Método de la Pseudoinversa}

\UseRawInputEncoding
%Lenguaje C++
\lstinputlisting[basicstyle=\normalsize\ttfamily\color{black}, commentstyle=\color{mygray}, frame=single, numbersep=5pt, numberstyle=\tiny\color{mygray}, keywordstyle=\color{mygreen}, showspaces=false, showstringspaces=false, stringstyle=\color{myorange}, tabsize=2, language=C++, caption={pseudoinversa}]{parte3/C++/pseudoinversa.cpp}
\inputencoding{utf8}
\newpage


\section{Tema 4: Polinomio de interpolación.}

\subsection{Método 1: Método de Lagrange}
\UseRawInputEncoding
%Lenguaje Python
\lstinputlisting[language=Python,
basicstyle=\ttfamily\normalsize,morekeywords={as,__init__,MyClass}, keywordstyle=\color{violet}\ttfamily, frame=single, commentstyle=\color{gray}\ttfamily, showstringspaces=false, caption={Lagrange}]{parte4/lagrange.py}
\inputencoding{utf8}

\newpage

\subsection{Método 2: Método de Diferencias Divididas de Newton}
\UseRawInputEncoding
%Lenguaje Python
\lstinputlisting[language=Python,
basicstyle=\ttfamily\normalsize,morekeywords={as,__init__,MyClass}, keywordstyle=\color{violet}\ttfamily, frame=single, commentstyle=\color{gray}\ttfamily, showstringspaces=false, caption={Diferencias Divididas de Newton}]{parte4/dd_newton.py}
\inputencoding{utf8}

\newpage

\subsection{Método 3: Trazador Cúbico Natural}
\UseRawInputEncoding
%Lenguaje Python
\lstinputlisting[language=Python,
basicstyle=\ttfamily\normalsize,morekeywords={as,__init__,MyClass}, keywordstyle=\color{violet}\ttfamily, frame=single, commentstyle=\color{gray}\ttfamily, showstringspaces=false, caption={Trazador Cubico Natural}]{parte4/traz_cubico.py}
\inputencoding{utf8}


\newpage

\subsection{Método 4: Cota de Error Polinomio de Interpolación}
%Lenguaje Python
\lstinputlisting[language=Octave,
basicstyle=\ttfamily\normalsize,morekeywords={as,__init__,MyClass}, keywordstyle=\color{violet}\ttfamily, frame=single, commentstyle=\color{gray}\ttfamily, showstringspaces=false, caption={Cota de Error Polinomio de Interpolación}]{parte4/cota_poly_inter.m}
\inputencoding{utf8}

\newpage

\subsection{Método 5: Cota de Error Trazador Cúbico Natural}
\UseRawInputEncoding
%Lenguaje Python
\lstinputlisting[language=Python,
basicstyle=\ttfamily\normalsize,morekeywords={as,__init__,MyClass}, keywordstyle=\color{violet}\ttfamily, frame=single, commentstyle=\color{gray}\ttfamily, showstringspaces=false, caption={Cota de Error Trazador Cubico Natural}]{parte4/cota_traz_cubico.py}
\inputencoding{utf8}

\newpage

\section{Tema 5: Integración Numérica.}

\subsection{Método 1: Regla del Trapecio y Cota de Error.}
%Lenguaje Octave
\lstinputlisting[language=Octave,
basicstyle=\ttfamily\normalsize,morekeywords={as,__init__,MyClass}, keywordstyle=\color{violet}\ttfamily, frame=single, commentstyle=\color{gray}\ttfamily, showstringspaces=false, caption={Regla del Trapecio y Cota de Error}]{parte5/trapecio.m}
\inputencoding{utf8}

\newpage

\subsection{Método 2: Regla de Simpson y Cota de Error}
\UseRawInputEncoding
%Lenguaje Python
\lstinputlisting[language=Python,
basicstyle=\ttfamily\normalsize,morekeywords={as,__init__,MyClass}, keywordstyle=\color{violet}\ttfamily, frame=single, commentstyle=\color{gray}\ttfamily, showstringspaces=false, caption={Regla de Simpson y Cota de Error}]{parte5/simpson.py}
\inputencoding{utf8}

\newpage

\subsection{Método 3: Regla Compuesta del Trapecio y Cota de Error.}
%Lenguaje Octave
\lstinputlisting[language=Octave,
basicstyle=\ttfamily\normalsize,morekeywords={as,__init__,MyClass}, keywordstyle=\color{violet}\ttfamily, frame=single, commentstyle=\color{gray}\ttfamily, showstringspaces=false, caption={Regla Compuesta del Trapecio y Cota de Error}]{parte5/trapecio_compuesto.m}
\inputencoding{utf8}

\newpage

\subsection{Método 4: Regla Compuesta de Simpson y Cota de Error}
\UseRawInputEncoding
%Lenguaje Python
\lstinputlisting[language=Python,
basicstyle=\ttfamily\normalsize,morekeywords={as,__init__,MyClass}, keywordstyle=\color{violet}\ttfamily, frame=single, commentstyle=\color{gray}\ttfamily, showstringspaces=false, caption={Regla Compuesta de Simpson y Cota de Error}]{parte5/simpson_compuesto.py}
\inputencoding{utf8}

\newpage

\subsection{Método 5: Cuadratura Gaussiana y Cota de Error}
\UseRawInputEncoding
\lstinputlisting[language=Octave,
basicstyle=\ttfamily\normalsize,morekeywords={as,__init__,MyClass}, keywordstyle=\color{violet}\ttfamily, frame=single, commentstyle=\color{gray}\ttfamily, showstringspaces=false, caption={Cuadratura Gaussiana y Cota de Errorr}]{parte5/cuad_gaussiana.m}
\inputencoding{utf8}

\newpage
\section{Tema 6: Diferenciación Numérica.}

\subsection{Método 1: Método de Euler.}

\UseRawInputEncoding
%Lenguaje Python
\lstinputlisting[language=Python,
basicstyle=\ttfamily\normalsize,morekeywords={as,__init__,MyClass}, keywordstyle=\color{violet}\ttfamily, frame=single, commentstyle=\color{gray}\ttfamily, showstringspaces=false, caption={Metodo de Euler}]{parte6/euler.py}
\inputencoding{utf8}



\newpage
\subsection{Método 2: Método Predictor-Corrector.}

\UseRawInputEncoding
%Lenguaje Python
\lstinputlisting[language=Octave,
basicstyle=\ttfamily\normalsize,morekeywords={as,__init__,MyClass}, keywordstyle=\color{violet}\ttfamily, frame=single, commentstyle=\color{gray}\ttfamily, showstringspaces=false, caption={Metodo Predictor-Corrector}]{parte6/predictor_corrector.m}
\inputencoding{utf8}

\newpage
\subsection{Método 3: Runge-Kutta de Orden 4.}

\UseRawInputEncoding
%Lenguaje Python
\lstinputlisting[language=Python,
basicstyle=\ttfamily\normalsize,morekeywords={as,__init__,MyClass}, keywordstyle=\color{violet}\ttfamily, frame=single, commentstyle=\color{gray}\ttfamily, showstringspaces=false, caption={Runge-Kutta de Orden 4}]{parte6/runge_kutta_4.py}
\inputencoding{utf8}

\newpage

\subsection{Método 4: Adam-Bashford de Orden 4.}

\UseRawInputEncoding
%Lenguaje Octave
\lstinputlisting[language=Octave,
basicstyle=\ttfamily\normalsize,morekeywords={as,__init__,MyClass}, keywordstyle=\color{violet}\ttfamily, frame=single, commentstyle=\color{gray}\ttfamily, showstringspaces=false, caption={Adam-Bashford de Orden 44}]{parte6/adam_bashford_4.m}
\inputencoding{utf8}


\newpage
https://www.overleaf.com/project/61155049b499c66a08272784
\section{Tema 7: Valores y Vectores Propios.}

\subsection{Método 1: Método de la Potencia.}

\UseRawInputEncoding
%Lenguaje Octave
\lstinputlisting[language=Octave,
basicstyle=\ttfamily\normalsize,morekeywords={as,__init__,MyClass}, keywordstyle=\color{violet}\ttfamily, frame=single, commentstyle=\color{gray}\ttfamily, showstringspaces=false, caption={Metodo de la Potencia}]{parte7/potencia.m}
\inputencoding{utf8}

\subsection{Método 2: Método de la Potencia Inversa.}

\UseRawInputEncoding
%Lenguaje C++
\lstinputlisting[language = C++,
basicstyle=\ttfamily\normalsize,morekeywords={as,__init__,MyClass}, keywordstyle=\color{violet}\ttfamily, frame=single, commentstyle=\color{gray}\ttfamily, showstringspaces=false, caption={Metodo de la Potencia Inversa}]{parte7/potencia_inversa.cpp}
\inputencoding{utf8}


\subsection{Método 3: Método QR.}

\UseRawInputEncoding
%Lenguaje Octave
\lstinputlisting[language = Python,
basicstyle=\ttfamily\normalsize,morekeywords={as,__init__,MyClass}, keywordstyle=\color{violet}\ttfamily, frame=single, commentstyle=\color{gray}\ttfamily, showstringspaces=false, caption={Metodo QR}]{parte7/metodo_qr.py}
\inputencoding{utf8}

\end{document}
